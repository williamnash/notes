\chapter{Number Theory}
\section{Misc}

\subsection{Geometric Series}
The geometric series is defined as 
\begin{align}
	\sum_{n=0}^{N-1} ar^n =a\Big( \frac{1-r^{N}}{1-r}\Big)
\end{align}
Which is only convergent for $r < 1$. To prove it, let's call whatever the value of the total series is equal to $s$, so we have
\begin{align}
	s = a + ar + ar^2 + ar^3 + ... + ar^{N-1}
\end{align}
Let's multiply it by $r$
\begin{align}
	rs =  ar + ar^2 + ar^3 + ... + ar^{N}
\end{align}
We recognize that the portion on the right is basically the same as the earlier expression for $s$, so let's plug in exactly what that is
\begin{align}
	rs = (s - a) + ar^N
\end{align}
Rearranging this, we find
\begin{align}
	(r-1)s = a(r^N - 1)
\end{align}
Dividing through, we see that
\begin{align}
	\sum_{n=0}^{N-1} ar^n = s = a\Big( \frac{1-r^{N}}{1-r}\Big)
\end{align}


\subsection{Completing the Square}
If we have some function that looks like
\begin{align}
ax^2 + bx + c
\end{align}
and we want it to look like
\begin{align}
d(x+e)^2 + f
\end{align}
We have that 
\begin{align}
ax^2 + bx + c = dx^2 + 2ed x + de^2 + f
\end{align}
Matching powers of $x$, we get that
\begin{align}
a &= d\\
b &= 2ed\\
c &= de^2+f
\end{align}
Solving these gives us
\begin{align}
d = a && e =\frac{b}{2a}  && f = c-\frac{b^2}{4a}
\end{align}




\subsection{Trigonometric Identities}

\begin{align}
\sin(\alpha\pm\beta) = \sin\alpha\cos\beta\pm\cos\alpha\sin\beta\\
\cos(\alpha\pm\beta) = \cos\alpha\cos\beta\mp\sin\alpha\sin\beta
\end{align}

With these identities we can derive the double angle formula, setting $\alpha = \beta$ so
\begin{align}
\cos2\alpha &= \cos^2\alpha -\sin^2\alpha\\
&= 2\cos^2\alpha-1
\end{align}
The Law of Cosines is 
\begin{align}
c^2 = a^2 + b^2 -2ab\cos\theta
\end{align}
Where $\theta$ is the angle between $a$ and $b$. Can remember this because reduces to the Pythagorean Theorem for $\theta = \pi/2$ and when $\theta = \pi$ we have
\begin{align}
c^2 = a^2 + b^2 +2ab = (a+b)^2
\end{align}
A much simpler way to recover this same identity is to draw out the vectors themselves
\begin{align}
	\textbf{a} + \textbf{c} = \textbf{b} \rightarrow \textbf{c} = \textbf{a} - \textbf{b}
\end{align}
Looking at the square of this, we have
\begin{align}
	c^2 &= a^2 + b^2 - 2\textbf{a}\cdot\textbf{b}\\
	&= a^2 + b^2 -2ab\cos\theta 
\end{align}

You can always remember the derivatives of a trigonometric function by looking at it's Taylor expansion, and taking the derivative of that
\begin{align}
	\sin x &= x - \frac{x^3}{3!} + \frac{x^5}{5!} + ...\\
	\cos x &= 1 - \frac{x^2}{2!} + \frac{x^4}{41} + ...
\end{align}
Taking the derivative of sin for instances gives us
\begin{align}
	\frac{d}{dx}\sin x &= \frac{d}{dx}\Big( x - \frac{x^3}{3!} + \frac{x^5}{5!} + ...\Big)\\
	&= 1 - \frac{x^2}{2!} + \frac{x^4}{41} + ...\\
	&= \cos x
\end{align}

\subsection{Polynomial Algebra}

Can find roots of polynomial equations with
$$
\begin{array}{rc@{}c@{}c@{}c@{}}
& -x^2 +& x + &2 & \\ \cline{2-4}
\multicolumn{1}{r|}{x+1} &-x^3+ & 0 + & 3x +&2\\
-&-x^2 -&x  &\\ \cline{2-3}
& 0 +& x^2 &\\ 
-&  & x^2 +&x\\ \cline{3-4}
& & 0 +&2x &\\ 
-& &  &2x +&2\\ \cline{4-5}
& & & &0\\
\end{array}
$$

\section{Combinatorics}
A critical combinatorics formula is 
\begin{align}
{{n}\choose{k}} = \frac{n!}{k!(n-k)!}
\end{align}
Which tells us how may ways we can pick $k$ objects out of a total of $n$ of them. Obviously we should have $n>k$, so the number is positive, which means we should have $n$ on the top.


\subsection{Euclidean Algorithm}

If you have two numbers that you don't know the prime factorization of, one can find their greatest common denominator (qcd) using this algorithm. 
\begin{enumerate}
\item Choose two numbers $N_1, N_2$ (e.g.`57, 27 `)
\item Subtract the smaller of the two from the larger $$max(N_1,N_2) - min(N_1,N_2)$$
    - 57 - 27 = 30
\item Continue to subtract $min()$
\end{enumerate}

\subsection{Modular Arthimetic}
